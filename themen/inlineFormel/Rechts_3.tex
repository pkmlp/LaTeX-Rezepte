
Der \LaTeX-Befehl \texttt{\char`\\bigskip} wird zwischen die Formel-Sätze und die Zwischenzeile (... oder auch ...) gesetzt um auch in der Textbox (\texttt{tcolorbox}) Abstand zu erhalten. Dieser Befehl hat nichts mit der eigentlichen  Problemstellung zu tun und ist ausserhalb der Textbox nicht notwendig, da der Abstand zwischen den Abschnitten (Paragraphen) in der Dokumentenklasse eingestellt werden kann.

Wie dieses Beispiel zeigt, sind dieser Form Grenzen gesetzt. \LaTeX\ setzt die Formeln so, dass die Zeilenabstände im Fliesstext gleich bleiben. Damit sind dieser Form natürlich Grenzen gesetzt. Brüche, Formeln mit Subscript oder Superscript funktionieren schnell nicht mehr zufriedenstellend. Komplexere Formeln werden besser freigestellt (siehe nächste Beispiele). 
