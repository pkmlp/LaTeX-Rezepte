
... ein Theorem (oder Lemma, oder Proposition, oder Satz, oder Korollar) in meinem Dokument.

Ein Satz oder Theorem ist in der Mathematik eine widerspruchsfreie logische Aussage, die mittels eines Beweises als wahr erkannt, das heisst, aus Axiomen, Definitionen und bereits bekannten Sätzen hergeleitet werden kann.

Ein Satz wird nach seiner Rolle, seiner Bedeutung oder seinem Kontext oft auch anders bezeichnet. Innerhalb eines Artikels oder einer Monografie (z. B. einer Dissertation oder einem Lehrbuch) verwendet man:

\begin{itemize}
\item Lemma (oder Hilfssatz) für eine Aussage, die nur im Beweis anderer Sätze im gleichen Werk verwendet wird und unabhängig davon keine Bedeutung hat,
\item Proposition für eine ebenfalls hauptsächlich lokal bedeutsame Aussage, etwa einen Hilfssatz, der in mehr als einem Beweis verwendet wird,
\item Satz (oder Theorem) für eine wesentliche Erkenntnis, die im Werk dargestellt wird, und
\item Korollar (oder Folgesatz) für eine triviale Folgerung, die sich aus einem Satz oder einer Definition ohne grossen Aufwand ergibt.
\end{itemize}

Die Einordnung eines Satzes in eine der oben genannten Kategorien ist subjektiv und hat keine Folgen für die Verwendung des Satzes. Viele Autoren verzichten auf den Begriff Proposition und setzen dafür Lemma oder Satz ein. Auch Korollar wird nicht immer von Satz unterschieden. Dagegen ist es durchaus üblich und für den Leser hilfreich, wenn reine Hilfssätze als solche erkennbar sind.

Sätze, die allgemein bekannt sind und in der Regel nicht mit der Originalquelle zitiert werden, tragen den Namen des Gegenstandes, über den sie eine Aussage machen oder den Namen des Urhebers oder beides. In diesem Zusammenhang werden auch die Begriffe Fundamentalsatz oder Hauptsatz (eines Gebiets der Mathematik) verwendet, und die Unterscheidung zwischen Satz und Lemma ist oft eher historisch gewachsen als durch Inhalt und Bedeutung bestimmt. Viele Beispiele solcher Namen finden sich in der Liste mathematischer Sätze.
