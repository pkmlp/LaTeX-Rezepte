
In der Präambel:

\begin{verbatim}

% Packages für Mathe (Symbolverzeichnis)
\usepackage{amsmath, amssymb, amsthm}

\end{verbatim}

\tcblower

Erfassen der verwendeten Symbole in einer separaten Datei.

\begin{verbatim}

\begin{table}[h!]
\noindent\begin{tabular}{@{}p{2cm}l}
$\mathbf{N}$        & Menge aller natürlichen Zahlen ohne die Null \\
$\mathbf{N}_{0}$    & Menge aller natürlichen Zahlen mit der Null \\
$\pi$               & Die Kreiszahl Pi \\
$\Omega$            & Der elektrische Widerstand Ohm \\
$\boldsymbol\alpha$ & Alpha, der erste Buchstabe des griechischen Alphabetes \\
\end{tabular}
\end{table}

\end{verbatim}

Im Dokument (an der Stelle, an der das Symbolverzeichnis erscheinen soll): 

\begin{verbatim}

% Einfügen einer Leerseite ohne Seitenummer
\newpage
\thispagestyle{empty}
\mbox{}

% Erstellen des Symbolverzeichnis
\newpage
\section*{Symbolverzeichnis}
\addcontentsline{toc}{section}{Symbolverzeichnis}

\begin{table}[h!]
\noindent\begin{tabular}{@{}p{2cm}l}
$\mathbf{N}$        & Menge aller natürlichen Zahlen ohne die Null \\
$\mathbf{N}_{0}$    & Menge aller natürlichen Zahlen mit der Null \\
$\pi$               & Die Kreiszahl Pi \\
$\Omega$            & Der elektrische Widerstand Ohm \\
$\boldsymbol\alpha$ & Alpha, der erste Buchstabe des griechischen Alphabetes \\
\end{tabular}
\end{table}


\end{verbatim}
