
In der Regel werden Texte mit einer Textverarbeitung erstellt. Für professionelle Texte reichen diese aber nicht. So werden an Diplomarbeiten oder Dissertationen höhere Ansprüche gestellt. \LaTeX~ist ein Textsatz-System, das diese Ansprüche erfüllt. Ob in Uni, Fachhochschulen oder höheren Fachschulen - früher oder später stolpern alle über \LaTeX.

Zugegeben: wer an eine Textverarbeitung wie bei OpenOffice, LibreOffice oder Microsoft Office gewohnt ist, wird den Einstieg in \LaTeX~sehr gewöhnungsbedürftig finden. Denn bei \LaTeX~kommt ein umfangreiches Textsatzsystem zum Einsatz, das für Bücher, wissenschaftliche Arbeiten und Artikel gedacht ist.  

\LaTeX~ist ein Softwarepaket, das die Benutzung des Textsatzsystems \TeX~durch die Verwendung von Makros vereinfacht. Das Textsatzsystem wurde einstmals in Hinblick auf die Verschlechterung des Schriftsatzes durch die Einführung von Computern erschaffen und findet heute insbesondere im naturwissenschaftlichen Bereich häufig Anwendung. Gründe dafür sind zum Beispiel die umfangreichen Möglichkeiten, mit denen das System den Satz von mathematischen Symbolen bis hin zu Musiknoten erlaubt. Aber auch im Hinblick auf die notwendigen Formalien wissenschaftlicher Arbeiten hilft \LaTeX~den Autoren dieser Arbeiten dabei, sich hauptsächlich auf den Text zu konzentrieren ohne ihn hierbei mit unnötigen optischen Fragestellungen zu konfrontieren, wie es typischerweise WYSIWYG-Software (z.B. Microsoft Word) tut. Dies allerdings bedeutet auch, dass die Hürden für den Einstieg in \LaTeX~zunächst etwas höher sind als in WYSIWYG-Software, da der unerfahrene Benutzer auf den ersten Blick nicht erkennt, wie sein Dokument letzendlich aussehen wird und wie er Einfluss auf die Gestaltung nehmen kann.

Bei \LaTeX~kümmert man sich nicht fortwährend um das Layout, das man lediglich bei Beginn eines neuen Dokumentes auswählt, sondern in erster Linie um den Text selbst. Früher gab es bei der Textproduktion eine Arbeitsteilung: Der Autor schrieb den Text und ein ausgebildeter Buchsetzer erstellte das Layout. Er kannte sich mit Typographie aus und hatte den Blick für die passende Gestaltung von Schriftgrössen, Zeilenlängen, Fussnoten, Literaturverzeichnissen, Tabellen oder Stichwortlisten. Dies ist wichtig, da ein gutes Layout das Lesen erleichtert und die Verständlichkeit erhöht.

\LaTeX~spielt seinen grossen Vorteil vor allem in Dokumenten mit mathematischen Formeln aus. Denn diese lassen sich mit den entsprechenden \LaTeX-Befehlen direkt in den Text eingeben und werden am Ende automatisch richtig gesetzt. \LaTeX~ist darum im wissenschaftlichen Bereich die erste Wahl für Arbeiten mit Formeln sowie Grafiken und professionellem Layout. 

Wo Licht ist, ist auch immer Schatten. Ein wesentlicher Nachteil soll ebenso Erwähnung finden: 

\begin{leftbar}
Die ersten Schritte mit \LaTeX~sind recht kompliziert, der Einstieg in WYSIWYG-Programme fällt zu Beginn erheblich leichter. Jedoch folgt auf den – zugegeben recht anspruchsvollen – Einstieg ein wesentlich einfacheres Arbeiten als mit erwähnter Konkurrenz.
\end{leftbar}

Die Einarbeitung ist also nicht so einfach wie bei einer normalen Textverarbeitung. Wer bisher nicht programmiert hat, mag beim ersten Blick auf \LaTeX~eingeschüchtert sein, wird dafür aber mit einem Dokument in professionellem Layout belohnt. Zur Erleichterung gibt es zudem hilfreiche Tools.

Das hier vorliegende Dokument enthält eine thematisch geordnete Sammlung verschiedener Problemstellungen und deren Lösung in \LaTeX. Die Beispiele sind so aufgebaut, dass jedes Beispiel ein einziges Thema behandelt. Somit ist gut sichtbar, welche Packages und welche Befehle für die konkrete Problemlösung benötigt werden. 

Grundkenntnisse in \LaTeX~sind für das Verständnis dieses Dokumentes von Vorteil. Haben Sie noch nie mit \LaTeX~gearbeitet, empfehle ich u.a. das Tutorial der TU Graz\footnote{\url{https://latex.tugraz.at/latex/tutorial}}.

Für die ersten praktischen Schritte mit \LaTeX~kann ich Overleaf\footnote{\url{https://de.overleaf.com/}} empfehlen, ein einfach bedienbarer Online-\LaTeX-Editor. Keine Installation notwendig, Versionskontrolle, Hunderte von \LaTeX-Vorlagen und ...
 