
%%%%%%%%%%%%%%%%%%%%%%%%%%%%%%%%%%%%%%%%%%%%%%%%%%%%%%%%%%
% Einstellungen für ganzes Dokument mit Sans-Serif-Schrift
%
%\documentclass[twoside, 
%               a4paper, 
%               10pt, 
%               parskip=full, 
%               sectionentrydots=true, 
%               listof=totoc, 
%               listof=entryprefix, 
%               numbers=endperiod]{scrartcl}
%\renewcommand{\familydefault}{\sfdefault}
%\usepackage[hidelinks]{hyperref}
%\urlstyle{sf}
%\renewcommand\textbullet{\ensuremath{\bullet}}
%
%
%%%%%%%%%%%%%%%%%%%%%%%%%%%%%%%%%%%%%%%%%%%%%%%%%%%%%
% Einstellungen für ganzes Dokument mit Serif-Schrift
%
\documentclass[twoside, 
               a4paper, 
              10pt, 
               parskip=full, 
               sectionentrydots=true, 
               listof=totoc, 
               listof=entryprefix,
               numbers=endperiod]{scrartcl}
\RedeclareSectionCommand[
  beforeskip=0\baselineskip,
  afterskip=.1\baselineskip]{section}
\RedeclareSectionCommand[
  beforeskip=0\baselineskip,
  afterskip=.1\baselineskip]{subsection}
\setkomafont{disposition}{\normalcolor\bfseries}
\usepackage[hidelinks]{hyperref}
%
%
%%%%%%%%%%%%%%%%%%%%%%%%%%%%%%%%%%%%%%%%%%%%%%%%%%%%%

% Einstellen der Seitenränder
\usepackage[top=1.5cm, bottom=2.5cm, left=3cm, right=1.5cm]{geometry}

% Packages für deutsche Umlaute
\usepackage[T1]{fontenc}
\usepackage[ngerman]{babel} 
\usepackage[utf8]{inputenc} 

% Packages für und Definiton der "miniSeiten"
\usepackage[most]{tcolorbox}
\newtcolorbox{miniSeite}[1][]{
  enhanced,
  arc=5pt,
  outer arc=5pt,
  colback=white,
  top=10pt,
  bottom=10pt,
  left=20pt,
  right=20pt,
  #1
}

% Package für Seitenlinien (siehe Vorwort)
\usepackage{framed}
\definecolor{isabelline}{rgb}{0.96, 0.94, 0.93}
\renewenvironment{leftbar}[1][\hsize]
{%
    \def\FrameCommand
    {
        {\color{gray}\vrule width 3pt}
        \hspace{0pt}
        \fboxsep=\FrameSep\colorbox{isabelline}
    }
    \MakeFramed{\hsize#1\advance\hsize-\width\FrameRestore}
}
{\endMakeFramed}

% Packages für Symbole und Formeln:
\usepackage{amsmath, amssymb}
\usepackage{amsthm}

% Packages für Programm-Listings:
\usepackage{listings}
\usepackage{color}
\definecolor{dkgreen}{rgb}{0,0.6,0}
\definecolor{mauve}{rgb}{0.58,0,0.82}

% Packages um PDF-Dateien einzubinden
\usepackage{pdfpages}

% Packages für Stichwortverzeichnis
\usepackage{makeidx}
\makeindex

% Definition eigener Befehle
\newcommand{\frob}{Frobeniushomomorphismus }



\begin{document}

% Theoreme, Sätze, ...
\newtheorem{satz}{Satz}
\newtheorem{theorem}{Theorem}[section]
\newtheorem*{behauptung}{Behauptung}


% Erstellen der Titelseite
\begin{center}
  \vspace*{5.5cm}
  \line(1,0){430}\\
  [5mm]
  \Huge \textbf{\LaTeX-Rezepte} \\
  [4mm]
  \line(1,0){430}\\
  \vspace{1.5cm}
  \LARGE\textbf{Problemstellungen und Lösungen}\\
  \vspace{1.5cm}
  \Large\textbf{Peter Kessler}\\  
  \Large\textbf{April 2019}\\  
  \vfill
\end{center}
 
\thispagestyle{empty}

% Einfügen einer Leerseite ohne Seitenummer
\newpage
\thispagestyle{empty}
\mbox{}

% Erstellen Vorwort
\newpage
\section*{Vorwort}
\addcontentsline{toc}{section}{Vorwort}
\input{inhalte/vorwort}

% Einfügen einer Leerseite ohne Seitenummer
\newpage
\thispagestyle{empty}
\mbox{}

% Erstellen Inhaltsverzeichnis
\newpage
\tableofcontents \label{toc} 



%%%%%%%%%%%%%%% --- Mathematik --- %%%%%%%%%%%%%%%%%%%

% Linke Seite - Problemstellung Inline Formeln
\newpage
\section{Mathematische Formeln}
\subsection{Inline Formeln}
{\textbf {Ich hätte gerne ...}}
\input{themen/inlineFormel/Links_1} 
\begin{miniSeite}[colbacktitle=black!35!white,title=Ausdruck]
\input{themen/inlineFormel/Links_2}
\end{miniSeite}

\textbf{Hinweis:} Für die Erstellung eines Stichwort-oder Namens-Verzeichnis muss ein spezielles \LaTeX-Werkzeug (MakeIndex) aufgerufen werden. 


% Rechte Seite - Lösung Inline Formeln
\newpage
{\textbf {Das erreiche ich mit ...}}

... mit der generischen Theoremumgebung von \LaTeX. 
 
\begin{miniSeite}[colbacktitle=black!35!white,title=\LaTeX-Code]
\input{themen/inlineFormel/Rechts_2}
\end{miniSeite}

Das ist im wesentlichen schon alles über die \LaTeX-Theorem-Umgebung. Viele weitere Optionen entnehmen Sie bitte der Dokumentation des Package. 




% Linke Seite - Problemstellung Abgesetzte Formeln
\newpage
\subsection{Abgesetzte Formeln}
{\textbf {Ich hätte gerne ...}}
\input{themen/abgesetzteFormel/Links_1} 
\begin{miniSeite}[colbacktitle=black!35!white,title=Ausdruck]
\input{themen/abgesetzteFormel/Links_2}
\end{miniSeite}

\textbf{Hinweis:} Für die Erstellung eines Stichwort-oder Namens-Verzeichnis muss ein spezielles \LaTeX-Werkzeug (MakeIndex) aufgerufen werden. 


% Rechte Seite - Lösung Abgesetzte Formeln
\newpage
{\textbf {Das erreiche ich mit ...}}

... mit der generischen Theoremumgebung von \LaTeX. 
 
\begin{miniSeite}[colbacktitle=black!35!white,title=\LaTeX-Code]
\input{themen/abgesetzteFormel/Rechts_2}
\end{miniSeite}

Das ist im wesentlichen schon alles über die \LaTeX-Theorem-Umgebung. Viele weitere Optionen entnehmen Sie bitte der Dokumentation des Package. 




% Linke Seite - Problemstellung abgesetzte, nummerierte und ausgerichtete  Formeln
\newpage
\subsection{Abgesetzte, nummerierte und ausgerichtete Formeln}
{\textbf {Ich hätte gerne ...}}
\input{themen/nummerierteFormel/Links_1} 
\begin{miniSeite}[colbacktitle=black!35!white,title=Ausdruck]
\input{themen/nummerierteFormel/Links_2}
\end{miniSeite}

\textbf{Hinweis:} Für die Erstellung eines Stichwort-oder Namens-Verzeichnis muss ein spezielles \LaTeX-Werkzeug (MakeIndex) aufgerufen werden. 


% Rechte Seite - Lösung abgesetzte, nummerierte und ausgerichtete  Formeln
\newpage
{\textbf {Das erreiche ich mit ...}}

... mit der generischen Theoremumgebung von \LaTeX. 
 
\begin{miniSeite}[colbacktitle=black!35!white,title=\LaTeX-Code]
\input{themen/nummerierteFormel/Rechts_2}
\end{miniSeite}

Das ist im wesentlichen schon alles über die \LaTeX-Theorem-Umgebung. Viele weitere Optionen entnehmen Sie bitte der Dokumentation des Package. 




% Linke Seite - Problemstellung abgesetzte, un-/nummerierte und ausgerichtete Formeln
\newpage
\subsection{Abgesetzte, un-/nummerierte und ausgerichtete Formeln}
{\textbf {Ich hätte gerne ...}}
\input{themen/unnummerierteFormel/Links_1} 
\begin{miniSeite}[colbacktitle=black!35!white,title=Ausdruck]
\input{themen/unnummerierteFormel/Links_2}
\end{miniSeite}

\textbf{Hinweis:} Für die Erstellung eines Stichwort-oder Namens-Verzeichnis muss ein spezielles \LaTeX-Werkzeug (MakeIndex) aufgerufen werden. 


% Rechte Seite - Lösung abgesetzte, un-/nummerierte und ausgerichtete Formeln
\newpage
{\textbf {Das erreiche ich mit ...}}

... mit der generischen Theoremumgebung von \LaTeX. 
 
\begin{miniSeite}[colbacktitle=black!35!white,title=\LaTeX-Code]
\input{themen/unnummerierteFormel/Rechts_2}
\end{miniSeite}

Das ist im wesentlichen schon alles über die \LaTeX-Theorem-Umgebung. Viele weitere Optionen entnehmen Sie bitte der Dokumentation des Package. 




%%%%%%%%%%%%%%% --- Theorem --- %%%%%%%%%%%%%%%%%%%

% Linke Seite - Problemstellung Theorem Formeln
\newpage
\section{Mathematische Behauptungen und Beweise}
\subsection{Theorem, Proof}

{\textbf {Ich hätte gerne ...}}
\input{themen/theorem/Links_1} 
\begin{miniSeite}[colbacktitle=black!35!white,title=Ausdruck]
\input{themen/theorem/Links_2}
\end{miniSeite}

\textbf{Hinweis:} Für die Erstellung eines Stichwort-oder Namens-Verzeichnis muss ein spezielles \LaTeX-Werkzeug (MakeIndex) aufgerufen werden. 


% Rechte Seite - Lösung Theorem Formeln
\newpage
{\textbf {Das erreiche ich mit ...}}

... mit der generischen Theoremumgebung von \LaTeX. 
 
\begin{miniSeite}[colbacktitle=black!35!white,title=\LaTeX-Code]
\input{themen/theorem/Rechts_2}
\end{miniSeite}

Das ist im wesentlichen schon alles über die \LaTeX-Theorem-Umgebung. Viele weitere Optionen entnehmen Sie bitte der Dokumentation des Package. 




%%%%%%%%%%%%%%% --- Listings --- %%%%%%%%%%%%%%%%%%%

% Linke Seite - Problemstellung Listings 
\newpage
\section{Programm Listings}
\subsection{Matlab, Python, Java, C++}

{\textbf {Ich hätte gerne ...}}
\input{themen/listings/Links_1} 
\begin{miniSeite}[colbacktitle=black!35!white,title=Ausdruck]
\input{themen/listings/Links_2}
\end{miniSeite}

\textbf{Hinweis:} Für die Erstellung eines Stichwort-oder Namens-Verzeichnis muss ein spezielles \LaTeX-Werkzeug (MakeIndex) aufgerufen werden. 


% Rechte Seite - Lösung Listings 
\newpage
{\textbf {Das erreiche ich mit ...}}

... mit der generischen Theoremumgebung von \LaTeX. 
 
\begin{miniSeite}[colbacktitle=black!35!white,title=\LaTeX-Code]
\input{themen/listings/Rechts_2}
\end{miniSeite}

Das ist im wesentlichen schon alles über die \LaTeX-Theorem-Umgebung. Viele weitere Optionen entnehmen Sie bitte der Dokumentation des Package. 




%%%%%%%%% --- Fussnoten --- %%%%%%%%%%

% Linke Seite - Problemstellung Fussnoten
\newpage
\section{Fussnoten}
\subsection{Fussnoten setzen}

{\textbf {Ich hätte gerne ...}}
\input{themen/fussnoten/Links_1} 
\begin{miniSeite}[colbacktitle=black!35!white,title=Ausdruck]
\input{themen/fussnoten/Links_2}
\end{miniSeite}

\textbf{Hinweis:} Für die Erstellung eines Stichwort-oder Namens-Verzeichnis muss ein spezielles \LaTeX-Werkzeug (MakeIndex) aufgerufen werden. 


% Rechte Seite - Lösung Fussnoten 
\newpage
{\textbf {Das erreiche ich mit ...}}

... mit der generischen Theoremumgebung von \LaTeX. 
 
\begin{miniSeite}[colbacktitle=black!35!white,title=\LaTeX-Code]
\input{themen/fussnoten/Rechts_2}
\end{miniSeite}

Das ist im wesentlichen schon alles über die \LaTeX-Theorem-Umgebung. Viele weitere Optionen entnehmen Sie bitte der Dokumentation des Package. 




%%%%%%%%% --- PDF Dateien einbinden --- %%%%%%%%%%

% Linke Seite - Problemstellung PDF Dateien einbinden
\newpage
\section{Einbinden von PDF-Dateien}
\subsection{PDF-Dateien}

{\textbf {Ich hätte gerne ...}}
\input{themen/pdfDateien/Links_1} 
\begin{miniSeite}[colbacktitle=black!35!white,title=Ausdruck]
\input{themen/pdfDateien/Links_2}
\end{miniSeite}

\textbf{Hinweis:} Für die Erstellung eines Stichwort-oder Namens-Verzeichnis muss ein spezielles \LaTeX-Werkzeug (MakeIndex) aufgerufen werden. 


% Rechte Seite - Lösung PDF Dateien einbinden 
\newpage
{\textbf {Das erreiche ich mit ...}}

... mit der generischen Theoremumgebung von \LaTeX. 
 
\begin{miniSeite}[colbacktitle=black!35!white,title=\LaTeX-Code]
\input{themen/pdfDateien/Rechts_2}
\end{miniSeite}

Die Eigenständigkeitserklärung soll auf einer ungeraden (linken) Seite des Dokumentes erscheinen.




%%%%%%%%% --- eigene Befehle --- %%%%%%%%%%

% Linke Seite - Problemstellung eigene Befehle
\newpage
\section{Eigene Befehle}
\subsection{Eigene Befehle definieren}

{\textbf {Ich hätte gerne ...}}
\input{themen/eigeneBefehle/Links_1} 
\begin{miniSeite}[colbacktitle=black!35!white,title=Ausdruck]
\input{themen/eigeneBefehle/Links_2}
\end{miniSeite}

\textbf{Hinweis:} Für die Erstellung eines Stichwort-oder Namens-Verzeichnis muss ein spezielles \LaTeX-Werkzeug (MakeIndex) aufgerufen werden. 


% Rechte Seite - Lösung eigene Befehle
\newpage
{\textbf {Das erreiche ich mit ...}}

... mit der generischen Theoremumgebung von \LaTeX. 
 
\begin{miniSeite}[colbacktitle=black!35!white,title=\LaTeX-Code]
\input{themen/eigeneBefehle/Rechts_2}
\end{miniSeite}

Das ist im wesentlichen schon alles über die \LaTeX-Theorem-Umgebung. Viele weitere Optionen entnehmen Sie bitte der Dokumentation des Package. 




%%%%%%%%% --- Stichworte & Stichwortverzeichnis --- %%%%%%%%%%

% Linke Seite - Problemstellung Stichworte
\newpage
\section{Stichworte \& Stichwortverzeichnis}
\subsection{Stichworte definieren}

{\textbf {Ich hätte gerne ...}}
\input{themen/stichworte/Links_1} 
\begin{miniSeite}[colbacktitle=black!35!white,title=Ausdruck]
\input{themen/stichworte/Links_2}
\end{miniSeite}

\textbf{Hinweis:} Für die Erstellung eines Stichwort-oder Namens-Verzeichnis muss ein spezielles \LaTeX-Werkzeug (MakeIndex) aufgerufen werden. 


% Rechte Seite - Lösung Stichworte 
\newpage
{\textbf {Das erreiche ich mit ...}}

... mit der generischen Theoremumgebung von \LaTeX. 
 
\begin{miniSeite}[colbacktitle=black!35!white,title=\LaTeX-Code]
\input{themen/stichworte/Rechts_2}
\end{miniSeite}

Das ist im wesentlichen schon alles über die \LaTeX-Theorem-Umgebung. Viele weitere Optionen entnehmen Sie bitte der Dokumentation des Package. 




% Linke Seite - Problemstellung Stichwortverzeichnis
\newpage
\subsection{Stichwortverzeichnis erstellen}
{\textbf {Ich hätte gerne ...}}
\input{themen/stichwortVerzeichnis/Links_1} 
\begin{miniSeite}[colbacktitle=black!35!white,title=Ausdruck]
\input{themen/stichwortVerzeichnis/Links_2}
\end{miniSeite}

\textbf{Hinweis:} Für die Erstellung eines Stichwort-oder Namens-Verzeichnis muss ein spezielles \LaTeX-Werkzeug (MakeIndex) aufgerufen werden. 


% Rechte Seite - Lösung Stichwortverzeichnis 
\newpage
{\textbf {Das erreiche ich mit ...}}

... mit der generischen Theoremumgebung von \LaTeX. 
 
\begin{miniSeite}[colbacktitle=black!35!white,title=\LaTeX-Code]
\input{themen/stichwortVerzeichnis/Rechts_2}
\end{miniSeite}

Das Stichwortverzeichnis soll auf einer ungeraden (linken) Seite des Dokumentes erscheinen.

\textbf{Zur Erinnerung:} Die Erstellung des Stichwortverzeichnisses erfodert ein spezielles \LaTeX-Werkzeug (MakeIndex). Damit alles korrekt erstellt wird, muss folgende Sequenz eingehalten werden:

\begin{enumerate}
\setlength\itemsep{-1em}
\item pdfLaTeX
\item pdfLaTeX
\item MakeIndex (nur wenn Stichwortverzeichnis vorhanden)
\item bibLaTeX (nur wenn Literaturverzeichnis vorhanden)
\item pdfLaTeX
\item pdfLaTeX
\item pdf ansehen
\end{enumerate}



%%%%%%%%% --- Stichwortverzeichnis ausgeben --- %%%%%%%%%%

% Einfügen einer Leerseite ohne Seitenummer
\newpage
\thispagestyle{empty}
\mbox{}

% Erstellen des Stichwortverzeichisses mit Eintrag in ToC
\newpage
\addcontentsline{toc}{section}{Stichwortverzeichnis}
\renewcommand{\indexname}{Stichwortverzeichnis \bigskip}\label{index}
\printindex



%%%%%%%%% --- Eigenständigkeitserklärung  --- %%%%%%%%%%

% Einfügen einer Leerseite ohne Seitenummer
\newpage
\thispagestyle{empty}
\mbox{}

% Einbinden der Eigenständigkeitserklärung mit Eintrag in ToC
\newpage
\addcontentsline{toc}{section}{Eigenständigkeitserklärung}
\section*{Eigenständigkeitserklärung
\footnote{Eingebundene PDF-Vorlage der ETH Zürich} 
\label{eigenstaendigkeitserklaerung}}
\begin{figure}
\includepdf[scale=0.8, pagecommand={}]{themen/pdfDateien/ethz_eigenstaendigkeitserklaerung.pdf}
\end{figure}



%%%%%%%%% --- Schlusswort --- %%%%%%%%%%

% Einfügen einer Leerseite ohne Seitenummer
\newpage
\thispagestyle{empty}
\mbox{}

% Schlusswort ohne Kapitelnummer mit Eintrag in ToC
\newpage
\addcontentsline{toc}{section}{Schlusswort}
\section*{Schlusswort}

Die hier vorliegenden \LaTeX-Rezepte sind entstanden aus Fragen die im Verlaufe der Zeit an mich gerichtet wurden. Fehlen für Sie interessante Themen und/oder Problemstellungen, senden Sie mir bitte eine Mail\footnote{peter.kessler@id.ethz.ch} und ich werde diese möglichst bald in dieses Dokument aufnehmen.

Haben Sie selber schon Lösungen/Beispiele zu Themen und/oder Problemstellungen die hier noch nicht enthalten sind, so können Sie mir diese sehr gerne zustellen, damit ich diese (natürlich mit Quellenangabe) in dieses Dokuemtn aufnehmen kann.

\LaTeX\ ist nie Liebe auf den ersten Blick. \LaTeX\ lernt man (mit zunehmender Erfahrung/Übung) lieben. Je mehr \LaTeX-Rezepte hier versammelt sind, desto leichter fällt es zukünftigen Studierenden sich in \LaTeX\ einzuarbeiten.


\end{document}
